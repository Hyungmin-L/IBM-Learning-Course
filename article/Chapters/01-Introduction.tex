\chapter[Qiskit in Classrooms: Quantum Mechanics]{Qiskit in Classrooms: Quantum Mechanics}
\label{cp:introduction}

\parindent0pt

\textit{Author: Hyungmin Lim}

\textit{Current Version: 0.0.1}

\textit{License: \LaTeX~Project Public License v1.3c}

\textit{Official Repository: \href{https://github.com/joseareia/ipleiria-thesis}{GitHub Repository}}

\vspace{.935em}

Welcome to the \textcolor{maincolor}{\textit{IBM Quantum Learning Course Guide}}! 

\section{Superposition with Qiskit}

In this section, we will find out the one of the most important quantum phenomenon-superposition. Through 


\section{Stern-Gelach measurements with Qiskit}

\subsection{Introduction}
\subsection{Classical coin}
\subsection{Quantum coin}
\subsection{The quantum revealed: an experiment in three dimensions}
\subsection{The quantum phase}
\subsection{Bloch sphere representation}
\subsubsection{Blochsphere examples}
Hello my freinds


\section{Exploring uncertainty with Qiskit}


\section{Bell's inequality with Qiskit}